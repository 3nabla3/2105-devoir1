% Alban Guyon (20206315)
% Université de Montréal
% 2105 - Informatique théorique
% May 23, 2022
% Used starting template from: https://guides.nyu.edu/LaTeX/sample-document

\documentclass{article}
\usepackage{amsmath}
\usepackage{algorithm}
\usepackage[noend]{algpseudocode}
\usepackage[margin=0.5in]{geometry}

\title{Introduction à l'informatique théorique -- 2005 devoir 1}
\author{Alban Guyon}
\date{May 23, 2022}

\begin{document}
    \maketitle
    \section{Questions}
    
    \textbf{1}
    Pour représenter une liste de nombres entiers en un seul nombre entier on utilise la notation suivante:
    \begin{equation}
    [a_1, a_2, a_3, ..., a_n, 0, 0, 0, ...] = 2^{a_1} * 3^{a_2} * 5^{a_3} * ... * p^{a_n}
    \end{equation}
    où les bases des exponents sont les nombres premiers.

    Du fait que chaque entier positif a une factorisation unique en nombres premiers, chaque liste
    sera associé à un nombre unique supérieur ou égal à 1.

    Une autre chose à noter est que toutes les listes sont infinis: une liste avec n éléments à réellement n
    éléments et une infinité de 0 qui suivent. En effet, les 0 à la fin d'une liste sont ignorés. 
    Par exemple, les listes [2, 3, 0, 3, 2, 0, 0] et [2, 3, 0, 3, 2] sont équivalantes.

    Dans le pseudo-code, on utilise la notation $P_i$ pour représenter le i-ème nombre premier.	
    Voici quelques fonctions sur les listes dans cette nouvelle représentation:

    \begin{algorithm}[H]
        \caption{Vide}\label{Vide}
        \begin{algorithmic}
        \Procedure{Vide()}{}
        \State $\textit{$r_0$} \gets \text{1}$
        \EndProcedure
        \end{algorithmic}
    \end{algorithm}

    \begin{algorithm}[H]
        \caption{EstVide?}\label{EstVide?}
        \begin{algorithmic}
        \Procedure{EstVide?($r_1$)}{}
        \If {$\textit{$r_1$} = \text{1}$}
            \State $\textit{$r_0$} \gets \text{TRUE}$
        \Else {}
            \State $\textit{$r_0$} \gets \text{FALSE}$
        \EndIf
        \EndProcedure
        \end{algorithmic}
    \end{algorithm}

    \begin{algorithm}[H]
        \caption{Dans?}\label{Dans?}
        \begin{algorithmic}
        \Procedure{Dans?($r_1, r_2$)}{}
        \State $r_1 \gets \text{FALSE}$
        \For {$i \gets 0$ \textbf{to} $r_1$} \Comment{on utilise $r_1$ pour la limite max car taille de la liste est forcément inférieure}
            \If {$\textit{$r_1$} \mod \text{$P_{i}^{r_2}$} = 0$ $\textbf{and not}$ $\textit{$r_0$}$}
                \State $\textit{N} \gets n / P_{i}^{r_2}$ \Comment{on enlève $r_1$ fois le même facteur premier}
                \If {$N \mod (P_{i}^{r_2}) \neq 0$} \Comment{on s'assure qu'il n'y a pas d'autre diviseurs}
                    \State $\textit{$r_1$} \gets \text{TRUE}$
                \EndIf
            \EndIf
        \EndFor
        \EndProcedure
        \end{algorithmic}
    \end{algorithm}
    
    \begin{algorithm}[H]
        \caption{Card}\label{Card}
        \begin{algorithmic}
        \Procedure{Card($r_1$)}{}
        \State $\textit{$r_0$} \gets \text{0}$
        \For {$i \gets 0$ \textbf{to} $r_1$}
            \If {$n \mod (P_{i}^{r_1}) = 0$}
                \State $\textit{$r_0$} \gets \textit{$i$} + 1$\Comment{l'indice du dernier element non nul + 1 est la taille}
            \EndIf
        \EndFor
        \EndProcedure
        \end{algorithmic}
    \end{algorithm}
    
    \begin{algorithm}[H]IND
        \caption{Ajouter}\label{Ajouter}
        \begin{algorithmic}
        \Procedure{Ajouter($r_2, r_1$)}{}
        \State $\textit{$SIZE$} \gets Card(r_1)$
        \State $\textit{$r_0$} \gets r_1 * P_{SIZE}^{r_2}$
        \EndProcedure
        \end{algorithmic}
    \end{algorithm}

    Pour la fonction Retirer, je vais d'abord définir les macos auxiliaires 
    $IndexDe(r_2, r_1)$ et $index(r_2, r_1)$:
    \newline
    $IndexDe(r_2, r_1)$ retourne l'indice de l'élément $r_2$ dans la liste $r_1$.
    \newline
    $index(r_2, r_1)$ retourne l'élément à l'indice $r_2$ dans la liste $r_1$

    \begin{algorithm}[H]
        \caption{IndexDe}\label{IndexDe}
        \begin{algorithmic}
        \Procedure{IndexDe($r_2, r_1$)}{}
        \State $\textit{$r_0$} \gets \text{$r_1$}$ \Comment{on initialise l'indice à $r_1$ qui signifie que l'élément n'est pas dans la liste}
        \State $\textit{FOUND} \gets \text{FALSE}$
        \For {$i \gets 0$ \textbf{to} $r_1$}
            \If {\textbf{not} \text{FOUND}}
                \State $\textit{COPY} \gets r_1$
                \State $\textit{FLAG} \gets \text{TRUE}$ \Comment{Pouvait-on enlever $r_2$ instaces de $P_{i}$?}
                \For {$j \gets 0$ \textbf{to} $r_2$}
                    \If {$\textit{COPY} \mod P_{j} = 0$}
                        \State $\textit{COPY} \gets \textit{COPY} / P_{j}$
                    \Else
                        \State $\textit{FLAG} \gets \text{FALSE}$
                    \EndIf
                \EndFor
                
                \If {$\textit{FLAG}$ $\textbf{and}$ $\textit{COPY} \mod P_{i} \neq 0$}\Comment {Si $P_{i}$ n'est plus un facteur de $COPY$}
                    \State $\textit{FOUND} \gets \text{TRUE}$
                    \State $\textit{$r_0$} \gets i$
                \EndIf
            \EndIf
        \EndFor
        \EndProcedure
        \end{algorithmic}
    \end{algorithm}

    \begin{algorithm}[H]
        \caption{Index}\label{RetIndexirer}
        \begin{algorithmic}
        \Procedure{Index($r_2, r_1$)}{}
        \State $\textit{$r_0$} \gets \text{0}$
        \State $\textit{$COPY$} \gets \text{$r_1$}$
        \For {$i \gets 0$ \textbf{to} $r_1$}\Comment{On compte le nombre de fois que l'on peut enlever $P_{r_2}$ de $r_1$}
            \If {$\textit{$COPY$} \mod P_{r_2} = 0$}
                \State $\textit{$COPY$} \gets \textit{$COPY$} / P_{r_2}$
                \State $\textit{$r_0$} \gets \textit{$r_0$} + 1$
            \EndIf
        \EndFor
        \EndProcedure
        \end{algorithmic}
    \end{algorithm}

    On peut maintenant utiliser ces macros dans la macro Retirer:
    
    \begin{algorithm}[H]
        \caption{Retirer}\label{Retirer}
        \begin{algorithmic}
        \Procedure{Retirer($r_2, r_1$)}{}
        \State $\textit{$r_0$} \gets \text{$r_1$}$\Comment{Par défaut, on retourne juste la liste sans modification}
        \State $\textit{$IND$} \gets \textit{IndexDe($r_2$, $r_1$)}$
        \If {$\textit{$IND$} \neq \text{$r_1$}$}\Comment{Si l'élément est dans la liste}
            \State $\textit{$COPY$} \gets \textit{$r_1$} / P_{IND}^{r_2}$ \Comment{On met l'élément que l'on veut enlever à 0}
            \For {$i \gets 0$ \textbf{to} $r_1$} \Comment{On décale toutes les valeurs suivantes un indice à gauche}
                \State $\textit{$VAL$} \gets \textit{Index(i + 1, COPY)}$
                \State $\textit{$COPY$} \gets \textit{$COPY$} / P_{i + 1}^{VAL}$
                \State $\textit{$COPY$} \gets \textit{$COPY$} * P_{i}^{VAL}$
            \EndFor
            \State $\textit{$r_0$} \gets COPY$
        \EndIf
        \EndProcedure
        \end{algorithmic}
    \end{algorithm}

    \begin{algorithm}[H]
        \caption{Inter}\label{Inter}
        \begin{algorithmic}
        \Procedure{Inter($r_1, r_2$)}{}
        \State $\textit{$r_0$} \gets \text{1}$
        \For {$i \gets 0$ \textbf{to} $r_1$}
            \State $\textit{$VAL$} \gets \textit{Index($i, r_1$)}$
            \If {$\textit{$Dans?(r_1, VAL)$}$ $\textbf{and not}$ $\textit{$Dans?(L, VAL)$}$} \Comment{Si l'élément est dans les deux listes et on l'a pas déjà écrit}
                \State $\textit{$r_0$} \gets \textit{Ajouter($VAL, r_0$)}$
            \EndIf
        \EndFor
        \EndProcedure
        \end{algorithmic}
    \end{algorithm}

    Pour représenter des ensembles d'ensembles, on peut utiliser une variante de l'algorithme précédent:
    chaque liste interne est d'abord remplacée par son nombre, et on obtiens une liste simple.
    Il suffit ensuite de calculer le nombre représentant la liste externe.

    Pour décoder, on décode le nombre représentant la liste externe, puis on décode chaque liste interne.
    \newline \newline
    \textbf{2}
    Montrez la croissance de $B_4(x)$ en fonction de $x$.
    \newline    
    Etape de base pour x = 1
    \begin{equation}
        B_4(1) = B_3(B_3(1)) = B_3(2^4-3) = B_3(2^{2^2}-3) = 2^{2^{2^2}} - 3
    \end{equation}
    Etape general pour x = x
    \begin{equation} 
        B_4(x) = B_3(B_3 ( ... B_3(1))) = B_3(B_3 ( ... B_3(2^2 - 3)))
    \end{equation} 
    avec x + 1 fois $B_3$ \newline
    Etape d'induction pour x = x + 1
    \begin{equation}
        B_4(x+1) = B_3(B_3 ( ... B_3 (2^{2^2} - 3)))
    \end{equation}
    avec x + 1 fois $B_3$

    On voit que chaque $B_3(x)$ ajoute un étage de plus à la tour de puissance de 2 dans l'argument. 
    On peut donc confirmer qu'il y a x + 3 étages de 2 dans la tour de puissance de 2.

    \textbf{3}
    Le programme qui peut faire augmenter la variable "a" le plus rapidement est le suivant:
    \begin{algorithm}[H]
        \caption{Fast}\label{Fast}
        \begin{algorithmic}
        \Procedure{Fast()}{}
        \State $\textit{inc(a)}$
        \State $\textit{inc(a)}$
        \State $\textit{inc(a)}$
        \State $\textit{inc(a)}$ \Comment{a = 4}
        \For {$i \gets 0$ \textbf{to} $a$}\Comment{Comme a $>$ 3, ca vaut le coup d'utiliser 3 lignes pour incrémenter a par plus que 3}
            \State $\textit{inc(a)}$
        \EndFor\Comment{a = 8}
        \For {$i \gets 0$ \textbf{to} $a$}
            \State $\textit{inc(a)}$
        \EndFor\Comment{a = 16}
        \For {$i \gets 0$ \textbf{to} $a$}
            \State $\textit{inc(a)}$
        \EndFor\Comment{a = 32}
        \For {$i \gets 0$ \textbf{to} $a$}
            \State $\textit{inc(a)}$
        \EndFor
        \dots
        \EndProcedure
        \end{algorithmic}
    \end{algorithm}

    On voit bien que la variable "a" à une croissance exponentielle de base 2.
    Le maximum que l'on peut augmenter la variable est de la doubler toutes les 3 lignes.
    En effet, après $l$ lignes, "a" ne peut pas excéder $2^{(l / 3)}$.
    Cela veut dire qu'un programme REPETERPASTROP ne peut jamais calculer $n^k$ si k est supérieur au
    nombre de lignes divisé par 3.

    \textbf{4}
    Le fait de remplacer l'opération d'assignation par l'opération  d'interchangement ne change pase la puissance de calcul
    L'opération d'assignation peut être remplacé par la macro suivante:
    
    \begin{algorithm}[H]
        \caption{assign}\label{assign}
        \begin{algorithmic}
        \Procedure{Assign(a, b)}{}
        \State $\textit{$a$} \leftarrow \rightarrow \textit{$c$}$
        \For {$0$ \textbf{to} $b$}
            \State $inc(a)$
        \EndFor
        \EndProcedure
        \end{algorithmic}
    \end{algorithm}

    La valeur c n'a pas été assigne donc il est égal à 0.
    De plus, on choisit un registre different pour c à chaque fois que l'on veut utiliser la macro.
    Comme on a un nombre illimité de registres, on peut utiliser la macro pour toutes les opérations d'assignation.
    Le nouveau programme sera forcément plus long que le programme original mais il ne sera pas moins puissant.

    \textbf{5}
    La boucle POUR n'est pas plus puissante que la boucle REPETER car on peut utiliser une boucle REPETER pour
    simuler une boucle POUR de la forme:\newline
    POUR i $\leftarrow$ a à b faire $<bloc>$ suivant

    \begin{algorithm}[H]
        \caption{PourSimulation}\label{PourSimulation}
        \begin{algorithmic}
        \Procedure{PourSimulation}{}
        \State $\textit{$i$} \gets a$
        \State $\textit{$rep$} \gets MOINS(b, a)$\Comment{On utilise la macro MOINS pour être sur que le resultat est $\geq 0$}
        \For {$0$ \textbf{to} $rep$}
            \State $<bloc>$
            \State $\textit{$i$} \gets i + 1$
        \EndFor
        \EndProcedure
        \end{algorithmic}
    \end{algorithm}

    Comme on peut remplacer toutes les boucles POUR par une boucle REPETER sans perdre de l'information, 
    on déduit que la boucle POUR n'est pas plus puissante que la boucle REPETER, ce n'est simplement qu'un sucre syntaxique.\newline \newline \newline
    Source code: https://github.com/3nabla3/2105-devoir1
\end{document} % This is the end of the document
